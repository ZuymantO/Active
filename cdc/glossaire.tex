%\newglossaryentry{}{name={},description={}}

\newglossaryentry{MI}{
name={MI},
description={moteur d'indexation}
}

\newglossaryentry{BI}{
name={BI},
description={base d'indexation}
}

\newglossaryentry{MR}{
name={MR},
description={moteur de recherche}
}

\newglossaryentry{parser}{
name={parser},
description={consiste \`{a} analyser un texte ainsi que sa structure syntaxique}
}

\newglossaryentry{regexp}{
name={expression r\'{e}guli\`{e}re},
description={cha\^{i}ne de caractère},
plural={expressions r\'{e}guli\`{e}res}
}

\newglossaryentry{arborescence}{
name={arborescence},
description={ensemble de fichiers contenus dans un r\'{e}pertoire donn\'{e}},
plural={arborescences}
}

\newglossaryentry{fichier}{
name={fichier},
description={fichier r\'{e}gulier ou r\'{e}pertoire},
plural={fichiers}
}

\newglossaryentry{daemon}{
name={daemon},
description={d\'{e}signe un type de programme informatique, un processus ou un ensemble de processus qui s'exécute en arri\`{e}re-plan plut\^{o}t que sous le contr\^{o}le direct d'un utilisateur}
}

\newglossaryentry{metadonnee}{
name={m\'{e}tadonn\'{e}e},
description={est une donn\'{e}e servant à d\'{e}finir ou d\'{e}crire une autre donn\'{e}e quel que soit son support},
plural={m\'{e}tadonn\'{e}es}
}

\newglossaryentry{socket}{
name={socket},
description={il s'agit d'un mod\`{e}le permettant la communication inter processus afin de permettre à divers processus de communiquer aussi bien sur une m\^{e}me machine qu’\`{a} travers un r\'{e}seau TCP/IP},
plural={sockets}
}

\newglossaryentry{tcp}{
name={TCP},
description={Transmission Control Protocol (litt\'{e}ralement, \enquote{Protocole de Contr\^{o}le de Transmissions}), est un protocole de transport fiable, en mode connect\'{e}. Dans le mod\`{e}le Internet, aussi appel\'{e} mod\`{e}le TCP/IP, TCP est situ\'{e} au niveau de la couche transport}
}

\newglossaryentry{dtd}{
name={DTD},
description={Document Type Definition (litt\'{e}ralement, \enquote{Définition de Type de Document}), est un document permettant de d\'{e}crire un mod\`{e}le de document SGML ou XML. Le mod\`{e}le est d\'{e}crit comme une grammaire de classe de documents : grammaire parce qu'il d\'{e}crit la position des termes les uns par rapport aux autres, classe parce qu'il forme une g\'{e}n\'{e}ralisation d'un domaine particulier, et document parce qu'on peut former avec un texte complet}
}

\newglossaryentry{xml}{
name={XML},
description={Extensible Markup Language (litt\'{e}ralement, \enquote{langage de balisage extensible}) est un langage informatique utilisant des balises (\enquote{>} et \enquote{<})}
}

\newglossaryentry{console}{
name={console},
description={logiciel qui émule le fonctionnement d'un terminal informatique}
}