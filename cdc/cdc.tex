\documentclass[a4paper,12pt]{report}

\usepackage[utf8]{inputenc}
\usepackage[T1]{fontenc}
\usepackage[francais]{babel}

\usepackage{geometry}
\usepackage{graphicx}
\usepackage{xcolor}
\usepackage{caption}

\usepackage{array}
\usepackage{tabularx}

\usepackage{titlesec}
\usepackage{titletoc}
\usepackage{fancyhdr}
\usepackage{titling}

\usepackage{amsmath}
\usepackage{amssymb}

\usepackage{listings}
\lstset{breaklines=true}
\lstset{language=XML}

\usepackage[pdftex=true]{hyperref}

\usepackage[babel=true]{csquotes}

\titleformat{\chapter}[hang]{\bf\huge}{\thechapter}{1pc}{}
\titlespacing{\chapter}{0pt}{-20pt}{20pt}

\renewcommand{\baselinestretch}{1.2}

\pagestyle{fancy}
\fancyhf{}
\fancyhead[L]{Cahier des charges}
\fancyhead[R]{Projet Génie logiciel 2012-2013}
\fancyfoot[L]{Université Paris Diderot Paris 7}
\fancyfoot[R]{\thepage}
\renewcommand{\footrulewidth}{0pt}
\renewcommand{\headrulewidth}{0pt}

\makeatletter
\let\ps@plain=\ps@fancy
\makeatother

\usepackage[xindy,nonumberlist,style=listgroup]{glossaries}
\makeglossaries
%\newglossaryentry{}{name={},description={}}

\newglossaryentry{MI}{
name={MI},
description={moteur d'indexation}
}

\newglossaryentry{BI}{
name={BI},
description={base d'indexation}
}

\newglossaryentry{MR}{
name={MR},
description={moteur de recherche}
}

\newglossaryentry{parser}{
name={parser},
description={consiste \`{a} analyser un texte ainsi que sa structure syntaxique}
}

\newglossaryentry{regexp}{
name={expression r\'{e}guli\`{e}re},
description={cha\^{i}ne de caractère},
plural={expressions r\'{e}guli\`{e}res}
}

\newglossaryentry{arborescence}{
name={arborescence},
description={ensemble de fichiers contenus dans un r\'{e}pertoire donn\'{e}},
plural={arborescences}
}

\newglossaryentry{fichier}{
name={fichier},
description={fichier r\'{e}gulier ou r\'{e}pertoire},
plural={fichiers}
}

\newglossaryentry{daemon}{
name={daemon},
description={d\'{e}signe un type de programme informatique, un processus ou un ensemble de processus qui s'exécute en arri\`{e}re-plan plut\^{o}t que sous le contr\^{o}le direct d'un utilisateur}
}

\newglossaryentry{metadonnee}{
name={m\'{e}tadonn\'{e}e},
description={est une donn\'{e}e servant à d\'{e}finir ou d\'{e}crire une autre donn\'{e}e quel que soit son support},
plural={m\'{e}tadonn\'{e}es}
}

\newglossaryentry{socket}{
name={socket},
description={il s'agit d'un mod\`{e}le permettant la communication inter processus afin de permettre à divers processus de communiquer aussi bien sur une m\^{e}me machine qu’\`{a} travers un r\'{e}seau TCP/IP},
plural={sockets}
}

\newglossaryentry{tcp}{
name={TCP},
description={Transmission Control Protocol (litt\'{e}ralement, \enquote{Protocole de Contr\^{o}le de Transmissions}), est un protocole de transport fiable, en mode connect\'{e}. Dans le mod\`{e}le Internet, aussi appel\'{e} mod\`{e}le TCP/IP, TCP est situ\'{e} au niveau de la couche transport}
}

\newglossaryentry{dtd}{
name={DTD},
description={Document Type Definition (litt\'{e}ralement, \enquote{Définition de Type de Document}), est un document permettant de d\'{e}crire un mod\`{e}le de document SGML ou XML. Le mod\`{e}le est d\'{e}crit comme une grammaire de classe de documents : grammaire parce qu'il d\'{e}crit la position des termes les uns par rapport aux autres, classe parce qu'il forme une g\'{e}n\'{e}ralisation d'un domaine particulier, et document parce qu'on peut former avec un texte complet}
}

\newglossaryentry{xml}{
name={XML},
description={Extensible Markup Language (litt\'{e}ralement, \enquote{langage de balisage extensible}) est un langage informatique utilisant des balises (\enquote{>} et \enquote{<})}
}

\newglossaryentry{console}{
name={console},
description={logiciel qui émule le fonctionnement d'un terminal informatique}
}

\usepackage{makeidx}
\makeindex

\begin{document}

\begin{titlepage}

\begin{flushright}
\includegraphics[scale=0.75]{"images/logo_p7"}
\end{flushright}

\fontfamily{phv}\selectfont
\vspace*{\stretch{0.5}}
\begin{center}\Huge\fontfamily{pag}\selectfont
Active
\end{center}

\vspace*{\stretch{2}}
\begin{tabularx}{\linewidth}{lXr}
Auteurs && Enseignants\\
NDJANDA MBIADA Jacques Charles (chef de projet) && ZACCHIROLI Stefano\\
LEBEAU Christophe && KESNER Delia\\
JOOSSEN Matthieu && SIGHIREANU Mihaela
\end{tabularx}

\end{titlepage}

\chapter*{Préface}
\section*{Document}
Ce document a pour but de décrire les besoins, les exigences, les contraintes et les limites que nous allons devoir respecter durant la réalisation de notre projet.

Il résume les tâches que le programme doit pouvoir effectuer.

Il est composé de d'une première partie ouverte à tout lecteur et d'une partie technique qui explique les spécificités du produit et est plutôt destinée à un public averti.

Il sert également de planning pour délimiter le temps nécessaire à la réalisation de chaque étape du projet et ce dans le but de le terminer à temps.

\section*{Versions}
Nous souhaitons présenter ci-dessous l'intérêt de chaque version de ce présent document. Les modifications et ajouts importantes.

\subsection*{Version 2}
Finalisation de la spécification en vu du début de développement.

Définition du langage de requête et liste des charges.

\subsection*{Version 2 : Beta $\leftarrow$ (HEAD)}
Définition du langage de requête et support de définition. Choix final de la structure global

Des langages et technologies utilisés.

\subsection*{Version 2 : Alpha}
Restructuration du document. Avec respect du modèle donné en cours

\subsection*{Version 1}
Brouillon complet avec éléments de spécification et de la structure global du projet

\glsaddall
\printglossaries

\tableofcontents

\chapter{Introduction}

\section{Les besoins}
Le progrès technologique est indéniable à notre époque, notamment en termes de  capacité de stockage des ordinateurs personnels. L'un des enjeux des systèmes pour PC est la gestion des données utilisateur\index{Utilisateur}, à savoir permettre à ces derniers d'accéder rapidement à une information datant de plusieurs semaines, mois ou années sans y perdre des heures de recherche.

La capacité de nos disques durs peut atteindre aujourd'hui des milliers de gigaoctets (Go) et nous pouvons donc stocker des milliards de \glspl{fichier}\index{Fichier} sur un ordinateur.

Un gestionnaire de système de \glspl{fichier}\index{Fichier} est un programme permettant d'accéder à ces données, d'y naviguer et de visualiser leur relation (quel \gls{fichier}\index{Fichier} appartient à quel dossier\index{Dossier} par exemple). Ce programme n'a pas pour rôle de trier mais de rechercher les \glspl{fichier}\index{Fichier} et c'est pourquoi, en l'état, l'utilisateur\index{Utilisateur} devra trier son disque dur à la main, si c'est ce qu'il souhaite. Bien que la plupart des systèmes d'exploitation\index{Syst\`{e}me d'exploitation} répandus disposent, en plus du gestionnaire de système de \glspl{fichier}\index{Fichier}, d'un programme d'indexation\index{Indexation}, nous souhaitons écrire un programme alternatif au programme d'indexation\index{Indexation} natif de Linux\index{Linux} / Unix\index{Unix} (nous n'envisageons pas, du moins dans un premier temps, de développer notre produit pour les systèmes d'exploitation Windows\index{Windows}) et pourquoi ne pas le surpasser sur le plan technique et esthétique.

La réalisation du projet ne doit nécessiter aucun fond d'investissement, user uniquement de technologies open sources ou libres d'utilisation sans contraintes particulières sur le produit final. Pour palier à tout dépense supplémentaire (notamment sur le matériel de travail), nous avons à disposition des machines sous différentes plates-formes (NT, Linux\index{Linux}, Unix\index{Unix}) connectées à internet en très haut débit. Tout module\index{Module} permettant la réalisation du projet est inexistant et devra alors être développé par l'équipe en charge du projet.
Le but est bien entendu de rendre le programme terminé, mais également de parvenir à produire ce dernier par une démarche professionnelle et de manière coopérative sous la direction d'un chef de projet.

\section{Brève description}
On peut distinguer facilement trois modules\index{Module} dans le projet :
\begin{itemize}
\item un moteur de recherche\index{Moteur de rechercher} : ce module\index{Module} est en charge notamment de la communication\index{Communication} avec l'utilisateur\index{Utilisateur}
\item une base d'indexation\index{Base d'indexation} : ce module\index{Module} est en charge de la sauvegarde des données
\item un moteur d'indexation\index{Moteur d'indexation} : ce module\index{Module} est en charge de la surveillance\index{Surveillance} des \glspl{fichier}\index{Fichier} sur une \gls{arborescence}\index{Arborescence} donnée.
\end{itemize}

\begin{center}
\includegraphics[scale=0.45]{"images/fonctionnement_global_d'active"}
\captionof{figure}{Fonctionnement global d'Active}
\label{fonctionnement-global}
\end{center}

\section{Objectifs du produit}
Le produit que nous souhaitons développer devra répondre à certaines normes de communications\index{Communication} définies avec d'autres projets afin d'établir une inter-opérabilité\index{Inter-op\'{e}rabilit\'{e}} au niveau des trois modules\index{Module} principaux, définis brièvement dans la section précédente.

Il devra être opérationnel sur plusieurs architectures matériels afin qu'un utilisateur\index{Utilisateur} puisse l'utiliser sur plusieurs systèmes d'exploitation, sans avoir à changer ses habitudes.

Le module\index{Module} \enquote{moteur d'indexation}\index{Moteur d'indexation} sera chargé d'analyser les modifications\index{Modification} sur l'\gls{arborescence}\index{Arborescence} surveillée et d'en informer le module\index{Module} \enquote{base d'indexation}\index{Base d'indexation} afin que celui-ci mette à jour les informations dans la base de données\index{Base de donn\'{e}es}, qui stocke les informations sur tous les \glspl{fichier}\index{Fichier} sauvegardés.

L'\gls{arborescence}\index{Arborescence} surveillée doit être modifiable par l'utilisateur\index{Utilisateur}, ce qui entraîne, à chaque changement, une nouvelle indexation\index{Indexation} de la nouvelle \gls{arborescence}\index{Arborescence} à surveiller.

Notre produit devra permettre à l'utilisateur\index{Utilisateur} de relancer une indexation\index{Indexation} complète à tout
moment afin de s'assurer du bon contenu de la base de données\index{Base de donn\'{e}es}. Bien évidement, si le
produit est utilisé sur un appareil portable, lors d'une telle demande le moteur d'indexation\index{Moteur d'indexation} devra prendre en compte l'autonomie restante afin de ne pas empêcher l'utilisateur\index{Utilisateur} de pouvoir utiliser son appareil. Dans ce cas un message de confirmation d'indexation\index{Indexation} immédiate devra être donné à l'utilisateur\index{Utilisateur}.

Nous souhaitons que l'utilisateur\index{Utilisateur} puisse faire une recherche avec un maximum d'arguments
afin de limiter au maximum les résultats, ce qui permet de retrouver plus facilement un \gls{fichier}\index{Fichier}.

L'affichage des résultats devra offrir la possibilité d'être triés.

\chapter{Charges d'utilisation}

\section{Généralité}

\subsection{Public visé}
Nous visons toute personne utilisant un ordinateur sur lequel notre programme est compatible. Notre programme a pour but de permettre la recherche d'un \gls{fichier}\index{Fichier} à l'aide de mots clés. Des options de recherche sont également disponible tels qu'une fourchette de date, le type de \gls{fichier}\index{Fichier}. Les mots clé peuvent concerné le nom de \gls{fichier}\index{Fichier} et/ou le contenu. Ainsi, si l'utilisateur\index{Utilisateur} ne se souvient que de quelques informations sur un de ses \glspl{fichier}\index{Fichier}, dont un minimum de mots il peut utiliser notre programme afin de tenter de retrouver son \gls{fichier}\index{Fichier}. Dans le cas où notre programme trouve un résultat (un ou plusieurs \glspl{fichier}\index{Fichier}) ceux-ci seront affichés à l'utilisateur\index{Utilisateur}.

\subsection{Pré-requis}
Au niveau système, il sera demandé à l'utilisateur\index{Utilisateur} d'avoir une machine tournant sous Linux\index{Linux}, GNU\index{GNU} ou Unix\index{Unix}, d'avoir installer au préalable les librairies\index{Librairie} pour \enquote{sqlite3}\index{sqlite3} ainsi que celles de \enquote{tinyxml}\index{tinyxml} afin de pouvoir utiliser notre programme correctement.

Une interface fonctionnant via \gls{console}\index{Console} sera proposée. Cette interface nécessite un minimum de notion d'utilisation des lignes de commandes. Toutefois, pour les autres utilisateurs\index{Utilisateur}, l'utilisation avec interface graphique est équivalente en terme de recherche.

\section{Spécifications système des utilisateurs}

\subsection{Cas d'utilisation en recherche}
Lorsque l'utilisateur\index{Utilisateur} lance le programme, il lui est demandé de donner un ou des mots clés et, si il le désire, de remplir des informations complémentaires (intervalles de temps pour la création\index{Cr\'{e}ation} et la modification\index{Modification}, type de \gls{fichier}\index{Fichier}).

Le requête utilisateur\index{Utilisateur} est alors analysée afin de vérifier si elle est correcte. Si c'est le cas, elle est alors transformée en une ou plusieurs requête \gls{xml}\index{XML} (conformément à la \gls{dtd}\index{DTD}, p.~\pageref{dtd_bi_mr_search}) en fonction de la recherche demandée. Par exemple, la recherche sur un mot contenu dans le fichier\index{Fichier}, régulier, et sur un autre correspondant au nom du \gls{fichier}\index{Fichier} impliquera deux requêtes \gls{xml}\index{XML}. Il en va de même en cas de conjonction (\enquote{et}) et de disjonction (\enquote{ou}).

Les requêtes \gls{xml}\index{XML}, contenant chacune un identifiant différent, sont alors envoyées à la base d'indexation\index{Base d'indexation} qui va les traiter et envoyer , pour chacune d'elle, une réponse sous forme \gls{xml}\index{XML} (conformément à la \gls{dtd}\index{DTD}, p.~\pageref{dtd_bi_mr_result}) contenant l'identifiant correspondant à la requête de recherche.

Ces résultats vont être alors analysés et concaténés en fonction de la recherche initiale (par exemple, pour les conjonctions une intersection des résultats sera faite) puis affichés à l'utilisateur\index{Utilisateur}.

Voir cas d'utilisation p.~\pageref{utilisation-recherche} et diagramme de séquence p.~\pageref{sequence-recherche}.

\subsection{Cas d'utilisation en indexation}
L'utilisateur\index{Utilisateur} peut à tout moment interagir avec le moteur d'indexation\index{Moteur d'indexation}. Les opérations permises sont :
\begin{itemize}
\item démarrer le \gls{daemon}\index{Daemon} (sur un répertoire précis ou non)
\item redémarrer le \gls{daemon}\index{Daemon} (sur un répertoire précis ou non)
\item arrêter le \gls{daemon}\index{Daemon}
\item lister les \glspl{fichier}\index{Fichier} surveillés
\item supprimer la surveillance\index{Surveillance} sur un \gls{fichier}\index{Fichier} (récursivement ou non)
\item tuer le \gls{daemon}\index{Daemon}
\end{itemize}

La communication\index{Communication} s'effectue via \gls{socket}.

Voir le diagramme de décision p.~\pageref{decision-mi}.

\chapter{Architecture du système}

\section{Vue d'ensemble}
Le projet que nous développons a pour but d'être compatible avec des machines équipée d'un système d'exploitation Linux\index{Linux}, GNU\index{GNU} ou Unix\index{Unix} sur lequel les librairies\index{Librairie} \enquote{sqlite3}\index{sqlite3} et \enquote{tinyxml} auront été installée au préalable (car nécessaires pour le bon fonctionnement du programme).

Les exécutables que nous produisons est issus de code C++ et Java.
Pour son fonctionnement, notre programme crée un base de données\index{Base de donn\'{e}es} et vérifie l'existence de celle-ci à chaque lancement. Cette base de données\index{Base de donn\'{e}es} sert à stocker les informations sur les \glspl{fichier}\index{Fichier} surveillés.

\section{Modularité}
Le programme regroupe trois principaux modules\index{Module}, à savoir la base d'indexation\index{Base d'indexation} (voir diagramme de le classes p.~\pageref{diagramme_classes_bi}), le moteur de recherche \index{Moteur de recherche} et le moteur d'indexation\index{Moteur d'indexation} (voir le diagramme de classes p.~\pageref{diagramme_classes_bi}). Ces trois modules\index{Module} peuvent être remplacés par d'autres modules\index{Module} fonctionnant avec le même mode de fonctionnement (voir l'inter-opérabilité\index{Inter-op\'{e}rabilit\'{e}} p.~\pageref{inter_operabilite}). Un quatrième module\index{Module} vient se rajouter à ces derniers, le module\index{Module} \enquote{common} comportant les outils nécessaires à plusieurs des trois modules\index{Module} précédemment cités (voir les diagrammes de classes p.~\pageref{common}).

\subsection{Moteur de recherche}
Ce module\index{Module} est celui qui est lancé par l'utilisateur\index{Utilisateur} que ce soit avec un interaction avec la \gls{console}\index{Console} ou via l'interface graphique que nous proposons.

Ce module\index{Module} est celui qui s'occupe de la partie recherche. Il gère l'interaction avec l'utilisateur\index{Utilisateur}. Il doit se connecter à la base d'indexation\index{Base d'indexation} à chaque nouvelle requête et fermer cette connexion lors de la réception des résultats. Il doit transformer la requête utilisateur\index{Utilisateur} en flux \gls{xml}\index{XML} afin de communiquer correctement avec la base d'indexation\index{Base d'indexation} et doit \enquote{\gls{parser}} le flux \gls{xml}\index{XML} donné  en retour par la base d'indexation\index{Base d'indexation}.

Pour un requête utilisateur\index{Utilisateur} plusieurs requêtes \gls{xml}\index{XML} peuvent être crées. En effet, la \gls{dtd}\index{DTD} ne prend en compte ni la recherche simultanée d'un contenu et d'un nom de \gls{fichier}\index{Fichier} pour les mots clés ni les opérations sur les expressions (tel que les conjonctions, disjonction et plus généralement les \glspl{regexp}). Chacune des requêtes envoyées à la base d'indexation\index{Base d'indexation} le sera avec un identifiant qui sera réutilisé pour la réponse afin de pour analyser correctement les résultats.

En fonction de la demande initiale de l'utilisateur\index{Utilisateur} les requêtes seront alors combinées afin que le résultat corresponde à la requête de l'utilisateur\index{Utilisateur}. Dans la fenêtre d'affichage des résultats, les données pourront être triées en fonction d'un critère (nom du \gls{fichier}\index{Fichier}, date de création\index{Cr\'{e}ation}, date de dernière modification\index{Modification}, type de \gls{fichier}\index{Fichier}, propriétaire, groupe).

\subsection{Moteur d'indexation}
Il s'agit là d'un programme qui doit être lancé lors du démarrage de la session de l'utilisateur\index{Utilisateur}, autrement dit un \gls{daemon}\index{Daemon}. Une connexion avec la base d'indexation\index{Base d'indexation}, qui doit avoir été lancée au préalable, va alors être ouverte afin de permettre l'envoie de message.

Via des outils de surveillance\index{Surveillance} du système, il va être attentif à toutes les modifications\index{Modification} intervenant sur l'\gls{arborescence}\index{Arborescence} surveillée et en informer la base d'indexation\index{Base d'indexation}. Ces modifications\index{Modification} peuvent être une création\index{Cr\'{e}ation}, une suppression\index{Suppression} ou encore la modification\index{Modification} d'un \gls{fichier}\index{Fichier} (voir diagramme de décision p.~\pageref{decision-daemon}).

Pour chaque modification\index{Modification}, un flux \gls{xml}\index{XML} (respectant la \gls{dtd}\index{DTD} correspondante, voir p.~\pageref{dtd_bi_mi}) va alors être envoyé à la base d'indexation\index{Base d'indexation}.

L'utilisateur\index{Utilisateur} peut, à tout moment, entrer en contact avec le moteur d'indexation\index{Moteur d'indexation} pour lui demander de changer l'\gls{arborescence}\index{Arborescence} à surveiller, lancer une indexation\index{Indexation} sur l'\gls{arborescence}\index{Arborescence} actuellement sous surveillance\index{Surveillance}.

En cas de changement d'\gls{arborescence}\index{Arborescence}, une demande de suppression\index{Suppression} (de la base de données) de tous les \glspl{fichier}\index{Fichier} actuellement surveillés est alors demandée.
Ce module\index{Module} doit prendre en compte l'autonomie restante lors d'une utilisation sur un ordinateur n'étant pas branché sur le secteur afin de ne pas vider l'autonomie de celui-ci. Le programme doit pouvoir alors se mettre en veille et signaler à l'utilisateur\index{Utilisateur} de relancer une indexation\index{Indexation} lorsque l'autonomie aura atteint un seuil raisonnable d'utilisation.

\subsection{Base d'indexation}
La base d'indexation\index{Base d'indexation} doit être lancée comme serveur\index{Serveur} de manière à pouvoir accepter deux clients\index{Client} : un moteur de recherche\index{Moteur de rechercher} et un moteur d'indexation\index{Moteur d'indexation}. Ces communications\index{Communication} doivent s'établir de la façon définie par l'inter-opérabilité\index{Inter-op\'{e}rabilit\'{e}} (voir p.~\pageref{inter_operabilite}). De ce fait, la base d'indexation doit être lancée au démarrage de la session afin que les clients puissent se connecter. La base d'indexation peut donc être considérée comme un \gls{daemon}\index{Daemon}

La base d'indexation\index{Base d'indexation} doit être capable de \gls{parser} le \gls{xml}\index{XML} de la \gls{dtd}\index{DTD}, définie p.~\pageref{dtd} et de générer les réponses de requêtes utilisateur\index{Utilisateur} en respectant scrupuleusement la \gls{dtd}\index{DTD}. L'opération a effectuer sur la base de données\index{Base de donn\'{e}es} est un \verb"SELECT" en fonction des arguments reçus.

Les opérations possibles a effectuer lors de la réception d'un requête venant du moteur d'indexation\index{Moteur d'indexation} sont :
\begin{itemize}
\item nouvelle entrée lors de la création\index{Cr\'{e}ation} d'un \gls{fichier}\index{Fichier}, avec toutes les dépendances nécessaires, à l'aide de \verb"INSERT"
\item mise à jours d'un champ pour une modification\index{Modification}, en utilisant \verb"UPDATE"
\item suppression\index{Suppression} d'un champ pour une suppression\index{Suppression}, avec \verb"DELETE"
\end{itemize}
Pour ces trois cas, la modification\index{Modification} sur la base de donnée\index{Base de donn\'{e}es} doit se faire de manière complète, c'est à dire sur toutes les entrées correspondantes dans chacune des bases concernées. Par exemple, la transformation d'un fichier\index{Fichier} de type texte en fichier\index{Fichier} autre doit impliquer la suppression du contenu de ce fichier dans la base d'indexation.

Nous avons décidé de pouvoir insérer dans cette base de données\index{Base de donn\'{e}es} le maximum d'informations sur les \gls{fichier}\index{Fichier} afin que la recherche utilisateur\index{Utilisateur} puisse être la plus sélective possible. Un schéma de la base de donnée\index{Base de donn\'{e}es} est disponible en annexe p.~\pageref{bdd}, dans lequel on peut voir les informations stockées.

Là encore, il faut que les informations reçues respectent la \gls{dtd}\index{DTD}.

En revanche, la \gls{dtd}\index{DTD} ne prévoit pas de réponse du moteur de recherche\index{Moteur de recherche} lors de la communication\index{Communication} avec le moteur d'indexation\index{Moteur d'indexation}.

\subsection{Inter-opérabilité}\label{inter_operabilite}
Notre projet a pour but d'être inter-opérable avec tous les autres projets ayant le même but et utilisant le même protocole de communication\index{Communication} entre les modules\index{Module} et la même \gls{dtd}\index{DTD} (voir annexe p.~\pageref{dtd}) pour le contenu de la communication\index{Communication}.

Pour chacune des deux communication\index{Communication} la base d'indexation\index{Base d'indexation} sert de serveur\index{Serveur} et les moteurs de recherche\index{Moteur de recherche} et d'indexation\index{Moteur d'indexation} sont les clients\index{Client}.

Pour chacune des communications\index{Communication} trois ports\index{Port} ont été choisis dans l'idée que si le premier est utilisé on passe au deuxième et si celui-ci aussi est occupé, on se connecte au troisième. Si les trois ports\index{Port} sont occupés, la communication\index{Communication} ne pourra alors pas s'établir. Les ports\index{Port} choisis pour la communication\index{Communication} sont :
\begin{itemize}
\item 40000, 40001 et 40002 pour la communication\index{Communication} entre la base d'indexation\index{Base d'indexation} et le moteur d'indexation\index{Moteur d'indexation}
\item 30000, 30001 et 30002 pour la communication\index{Communication} entre la base d'indexation\index{Base d'indexation} et le moteur de recherche\index{Moteur de recherche}
\end{itemize}
Dans les deux cas les communications\index{Communication} se font en \gls{tcp}\index{TCP}.
Cette inter-opérabilité\index{Inter-op\'{e}rabilit\'{e}} a pour avantage de pouvoir faire évoluer les trois modules\index{Module} de manière indépendante tout en gardant l'intégrité du produit à partir du moment où les normes sont respectées.

\chapter{Spécifications du système}

\section{Au niveau du moteur d'indexation}
Il doit pouvoir scanner à tout moment le système (ou du moins la partie de l'\gls{arborescence}\index{Arborescence} surveillée) et ce de manière intelligente pour faire une mise à jour de la base de données\index{Base de donn\'{e}es}.\\
Il doit pouvoir afficher les \glspl{fichier}\index{Fichier} surveillés ainsi que sa capacité maximale de surveillance\index{Surveillance}, mais également ajouter ou enlever des \gls{fichier}\index{Fichier} de la surveillance\index{Surveillance}.\\
En attente d'événement (modification\index{Modification}, création\index{Cr\'{e}ation}, suppression\index{Suppression}) il doit attendre sans bloquer. Il doit stocker tous les événement en attendant de les stocker et les supprimer de sa liste une fois la requête envoyée à la base d'indexation\index{Base d'indexation} (afin de ne pas répéter l'événement).\\
Il doit accepter qu'un utilisateur\index{Utilisateur} se connecte pour communiquer avec lui.

Afin de fonctionner correctement sous toutes les architectures que nous visons, nous le munissons d'un sous module\index{Module} \enquote{ANotify} qui s'occupe d'utiliser les fonctions système de surveillance\index{Surveillance} sur les \glspl{fichier}\index{Fichier} propre au système d'exploitation.\\
Il doit pouvoir gérer les cas de déconnexion (brutales ou non), en se remémorant de toutes les informations à envoyer.\\
Il doit pouvoir gérer les mauvaises requêtes utilisateur\index{Utilisateur} (non conforme à la \gls{dtd}\index{DTD}). Toutes les requêtes crées par le moteur d'indexation\index{Moteur d'indexation} doivent être conforment à la \gls{dtd}\index{DTD}.

Un diagramme de classes du moteur d'indexation est fourni p.~\pageref{diagramme_classes_mi}.

\section{Au niveau du moteur de recherche}
Le moteur de recherche\index{Moteur de recherche} ne doit pas s'arrêter brutalement. Toutes les composants de l'interface graphique doivent s'afficher correctement et cela à tout moment (notamment lors de l'affichage des résultats).\\
Il doit pouvoir gérer les mauvaises requêtes utilisateur\index{Utilisateur} et les réponses des la base d'indexation\index{Base d'indexation} mal formées (non conforme à la \gls{dtd}\index{DTD}). Toutes les requêtes crées par le moteur de recherche\index{Moteur de recherche} doivent être conforment à la \gls{dtd}\index{DTD}.

Les coupures de connexions avec la base d'indexation\index{Base d'indexation} doivent être gérées de manière correcte. L'utilisateur\index{Utilisateur} doit être averti de la coupure de connexion et le choix doit lui être laissé d'arrêter ou non la recherche (qui reste en attente de connexion jusqu'à nouvelle connexion).\\
En cas de problème de connexion il faut empêcher l'utilisateur\index{Utilisateur} de faire une nouvelle recherche afin de ne pas bloquer toute l'application et à prendre de l'espace mémoire inutilement.

Un diagramme de classes du moteur de recherche est fourni p.~\pageref{diagramme_classes_mr}.

\section{Au niveau de la base d'indexation}
La base d'indexation\index{Base d'indexation} ne doit pas recréer la base de données\index{Base de donn\'{e}es} à chaque lancement.\\
La base d'indexation\index{Base d'indexation} ne doit en aucun cas se déconnecter sans avoir répondu à toutes les requêtes qu'elle a reçu.\\
Elle doit pouvoir gérer les requêtes des moteur de recherche\index{Moteur de recherche} et d'indexation\index{Moteur d'indexation} ne respectant pas la \gls{dtd}\index{DTD}. Toutes les requêtes crées par le moteur de recherche\index{Moteur de recherche} doivent être conforment à la \gls{dtd}\index{DTD}.

En cas de coupure de connexion avec un (ou plusieurs) des moteurs, toutes les réponses à envoyer doivent être stockées pendant une certaine durée dans l'attente que la connexion se refasse.

Un diagramme de classes de la base d'indexation est fourni p.~\pageref{diagramme_classes_bi}.

\chapter{Annexes}

\section{DTD}\label{dtd}

\subsection{Base d'indexation $\leftrightarrow$ Moteur d'indexation}\label{dtd_bi_mi}
\begin{lstlisting}[frame=single]
<!-- REGLE GENERALE Chaque fichier Xml doit contenir la ligne suivante :
     <?xml version="1.0" encoding="UTF-8"?>
-->

<!ELEMENT INDEXATION (RENOMMAGES?, MODIFICATIONS?, SUPPRESSIONS?, CREATIONS?)>
<!-- comme convenue toutes les balises de l'indexation sont facultatives -->

<!ELEMENT RENOMMAGES (FICHIERRENOMME)*>
<!ELEMENT MODIFICATIONS (FICHIERMODIFIE)*>
<!ELEMENT SUPPRESSIONS (FICHIERSUPPRIME)*>
<!ELEMENT CREATIONS (FICHIERCREE)*>

<!-- un fichier renomme n'est pas necessairement un fichier modifie -->
<!ELEMENT FICHIERRENOMME (PATH, NEWPATH)>
<!-- un fichier modife necessite une re-indexation -->
<!ELEMENT FICHIERMODIFIE (PATH, DATEMODIFICATION, TAILLE, PROPRIETAIRE, GROUPE, PERMISSIONS, INDEXAGE, NEWPATH?)>
<!ELEMENT FICHIERSUPPRIME (PATH)>
<!ELEMENT FICHIERCREE (PATH, format, DATECREATION, TAILLE, PROPRIETAIRE, GROUPE, PERMISSIONS, INDEXAGE)>

<!-- les meta-donnees -->
<!ELEMENT PATH (#PCDATA)>
<!ELEMENT format (#PCDATA)>
<!ELEMENT DATECREATION (#PCDATA)>
<!ELEMENT DATEMODIFICATION (#PCDATA)>
<!ELEMENT TAILLE (#PCDATA)>
<!ELEMENT PROPRIETAIRE (#PCDATA)>
<!ELEMENT GROUPE (#PCDATA)>
<!ELEMENT PERMISSIONS (#PCDATA)>
<!ELEMENT INDEXAGE (MOT*)>
<!ELEMENT MOT (#PCDATA)>
	<!ATTLIST MOT frequence CDATA #REQUIRED>
<!ELEMENT NEWPATH (#PCDATA)>

<!-- les id's seront utlises pour la tracabilitee et la detection d'eventuel erreurs, elle sont tout de meme facultatives -->

<!ATTLIST RENOMMAGES id CDATA #IMPLIED>
<!ATTLIST MODIFICATIONS id CDATA #IMPLIED>
<!ATTLIST SUPPRESSIONS id CDATA #IMPLIED>
<!ATTLIST CREATIONS id CDATA #IMPLIED>
\end{lstlisting}

\newpage
\subsection{Base d'indexation $\leftrightarrow$ Moteur de recherche}

\subsubsection{Rechercher}\label{dtd_bi_mr_search}
\begin{lstlisting}[frame=single]
<!-- Description de la requete de recherche -->
<!ELEMENT SEARCH (WORD, CONTENT?, PATHDIR?, PERM?, EXTENSION?, TIMESLOT?)>
	<!ATTLIST SEARCH id CDATA #REQUIRED>
<!-- Le mot a rechercher -->
<!ELEMENT WORD (#PCDATA)>
<!-- Un booleen qui dit si l'on fait une recherche de contenu (true) ou une recherche sur les nom de fichier (false) -->
<!ELEMENT CONTENT (#PCDATA)>
<!-- Le nom du fichier a partir duquel
     on recherche -->
<!ELEMENT PATHDIR (#PCDATA)>
<!-- Les permissions du fichier a chercher -->
<!ELEMENT PERM (#PCDATA)>
<!-- L'extension des fichiers a chercher -->
<!ELEMENT EXTENSION (#PCDATA)>
<!-- Intervalle de temps -->
<!ELEMENT TIMESLOT (BEGIN, END)>
<!-- Debut de l'intervalle -->
<!ELEMENT BEGIN (DAY, MONTH, YEAR)>
<!-- Fin de l'intervalle -->
<!ELEMENT END (DAY, MONTH, YEAR)>
<!-- Le jour -->
<!ELEMENT DAY (#PCDATA)>
<!-- Le mois -->
<!ELEMENT MONTH (#PCDATA)>
<!-- L'annee -->
<!ELEMENT YEAR (#PCDATA)>
\end{lstlisting}

\subsubsection{Résultat}\label{dtd_bi_mr_result}
\begin{lstlisting}[frame=single]
<!-- Balise contenant les resultats de la requete search -->
<!ELEMENT RESULT (FILE*)>
	<!ATTLIST RESULT id CDATA #REQUIRED>
<!-- Balise file correspond a un fichier resultat donc n balises files = n resultats -->
<!ELEMENT FILE (NAME, PATH, PERM, SIZE, LASTMODIF?, PROPRIO?)>
<!-- Le nom du fichier -->
<!ELEMENT NAME (#PCDATA)>
<!-- Le chemin complet du fichier -->
<!ELEMENT PATH (#PCDATA)>
<!-- Les permissions du fichier -->
<!ELEMENT PERM (#PCDATA)>
<!-- La taille du fichier -->
<!ELEMENT SIZE (#PCDATA)>
<!-- La date de derniere modification du fichier -->
<!ELEMENT LASTMODIF (#PCDATA)>
<!-- Le proprietaire du fichier -->
<!ELEMENT PROPRIO (#PCDATA)>
\end{lstlisting}

\section{Moteur de recherche}

\subsection{Cas d'utilisation}\label{utilisation-recherche}
\begin{center}
\includegraphics[scale=0.5]{"images/cas_utilisation_recherche"}
\captionof{figure}{Cas d'utilisation en recherche}
\end{center}

\subsection{Diagramme de séquence}\label{sequence-recherche}
\begin{center}
\includegraphics[scale=0.41]{"images/sequence_util"}
\captionof{figure}{Diagramme de séquence en recherche}
\end{center}

\section{Moteur d'indexation}

\subsection{Diagramme de classes}\label{diagramme_classes_mi}
\begin{center}
\includegraphics[scale=0.4]{"images/diagramme_classes_mi"}
\captionof{figure}{Diagramme de classes du moteur d'indexation}
\end{center}

\newpage
\subsection{Côté daemon}\label{decision-daemon}
\begin{center}
\includegraphics[scale=0.37]{"images/decision_mi"}
\captionof{figure}{Diagramme de décision du moteur d’indexation au niveau du daemon}
\end{center}

\newpage
\subsection{Côté utilisateur}\label{decision-mi}
\begin{center}
\includegraphics[scale=0.45]{"images/decision_util_mi"}
\captionof{figure}{Diagramme de décision du moteur d’indexation en interaction avec l'utilisateur}
\end{center}

\section{Base d'indexation}

\subsection{Schéma de la base de données}\label{bdd}
\begin{center}
\includegraphics[scale=0.37]{"images/bdd"}
\captionof{figure}{Schéma de la base de données}
\end{center}

\subsection{Diagramme de classes}\label{diagramme_classes_bi}
\begin{center}
\includegraphics[scale=0.4]{"images/diagramme_classes_bi"}
\captionof{figure}{Diagramme de classes de la base d'indexation}
\end{center}

\section{Common}\label{common}

\subsection{Diagramme de classes pour \enquote{AQuery}}
\begin{center}
\includegraphics[scale=0.35]{"images/diagramme_classes_aquery"}
\captionof{figure}{Diagramme de classes de la partie relative à \enquote{AQuery}}
\end{center}

\subsection{Diagramme de classes pour la génération du XML}
\begin{center}
\includegraphics[scale=0.4]{"images/diagramme_classes_xmlgeneration"}
\captionof{figure}{Diagramme de classes de la partie relative à la génération XML}
\end{center}

\newpage
\subsection{Diagramme de classes pour le parser XML}
\begin{center}
\includegraphics[scale=0.3]{"images/diagramme_classes_xmlparser"}
\captionof{figure}{Diagramme de classes de la partie relative au parser XML}
\end{center}

\clearpage
\addcontentsline{toc}{chapter}{Index}
\printindex

\end{document}
