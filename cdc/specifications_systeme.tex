\chapter{Spécifications du système}

\section{Au niveau du moteur d'indexation}
Mot(s) du glossaire utilisés : \gls{arborescence}, \gls{dtd}, \gls{fichier}.

Il doit pouvoir scanner à tout moment le système (ou du moins la partie de l'arborescence\index{Arborescence} surveillée) et ce de manière intelligente pour faire une mise à jour de la base de données\index{Base de donn\'{e}es}.\\
Il doit pouvoir afficher les fichiers\index{Fichier} surveillés ainsi que sa capacité maximale de surveillance\index{Surveillance}, mais également ajouter ou enlever des fichier\index{Fichier} de la surveillance\index{Surveillance}.\\
En attente d'événement (modification\index{Modification}, création\index{Cr\'{e}ation}, suppression\index{Suppression}) il doit attendre sans bloquer. Il doit stocker tous les événement en attendant de les stocker et les supprimer de sa liste une fois la requête envoyée à la base d'indexation\index{Base d'indexation} (afin de ne pas répéter l'événement).\\
Il doit accepter qu'un utilisateur\index{Utilisateur} se connecte pour communiquer avec lui.\\
Dans le cas de création, modification de fichier texte ou renommage vers un format textuel le moteur d'indexation\index{Moteur d'indexation} doit lancer l'indexation sur ce fichier. Il s'agit là de lire tous les mots contenus dans le fichiers avec leur occurrence afin de calculer leur taux d'apparition dans le fichier. Les mots apparaissant avec un taux d'au moins 5\% seront alors indexés dans la base de données\index{Base de donn\'{e}es}. A cette règles échappent les mots usuels (voir les \textit{stop list} française et anglaise p.~\pageref{stop_list}.

Afin de fonctionner correctement sous toutes les architectures que nous visons, nous le munissons d'un sous module\index{Module} \enquote{ANotify} qui s'occupe d'utiliser les fonctions système de surveillance\index{Surveillance} sur les fichiers\index{Fichier} propre au système d'exploitation.\\
Il doit pouvoir gérer les cas de déconnexion (brutales ou non), en se remémorant de toutes les informations à envoyer.\\
Il doit pouvoir gérer les mauvaises requêtes utilisateur\index{Utilisateur} (non conforme à la DTD\index{DTD}). Toutes les requêtes crées par le moteur d'indexation\index{Moteur d'indexation} doivent être conforment à la DTD\index{DTD}.

Un diagramme de classes du moteur d'indexation est fourni p.~\pageref{diagramme_classes_mi}.

\section{Au niveau du moteur de recherche}
Mot(s) du glossaire utilisés : \gls{console}, \gls{dtd}, \gls{parser}, \gls{xml}.

Le moteur de recherche\index{Moteur de recherche} ne doit pas s'arrêter brutalement. Toutes les composants de l'interface graphique doivent s'afficher correctement et cela à tout moment (notamment lors de l'affichage des résultats).\\
Il doit se connecter à la base d'indexation\index{Base d'indexation} à chaque nouvelle requête et fermer cette connexion lors de la réception des résultats. Il doit transformer la requête utilisateur\index{Utilisateur} en flux XML\index{XML}, ce qui peut correspondre à plusieurs flux, afin de communiquer correctement avec la base d'indexation\index{Base d'indexation} et doit \enquote{parser} le flux XML\index{XML} donné  en retour par la base d'indexation\index{Base d'indexation}.\\
Il doit pouvoir gérer les mauvaises requêtes utilisateur\index{Utilisateur} et les réponses des la base d'indexation\index{Base d'indexation} mal formées (non conforme à la DTD\index{DTD}). Toutes les requêtes crées par le moteur de recherche\index{Moteur de recherche} doivent être conforment à la DTD\index{DTD}.\\
Le client doit pouvoir se connecter directement avec la moteur de recherche via console\index{Console} ou via l'interface graphique fournie. L'interface graphique doit être au maximum indépendante du système.\\
L'utilisation de l'interface graphique fait appel au même programme que l'utilisateur doit utiliser afin de faire une requête. L'interface graphique doit donc pouvoir exécuter un programme externe et récupérer la sortie de celui-ci. L'interface graphique et l'utilisateur doivent respecter le langage définit p.~\pageref{langage_req} afin que la requête soit valide.

Les coupures de connexions avec la base d'indexation\index{Base d'indexation} doivent être gérées de manière correcte. L'utilisateur\index{Utilisateur} doit être averti de la coupure de connexion et le choix doit lui être laissé d'arrêter ou non la recherche (qui reste en attente de connexion jusqu'à nouvelle connexion).\\
En cas de problème de connexion il faut empêcher l'utilisateur\index{Utilisateur} de faire une nouvelle recherche afin de ne pas bloquer toute l'application et à prendre de l'espace mémoire inutilement.

Un diagramme de classes du moteur de recherche est fourni p.~\pageref{diagramme_classes_mr}.

% pour une meilleure mise en page
\newpage

\section{Au niveau de la base d'indexation}
Mot(s) du glossaire utilisés : \gls{dtd}, \gls{fichier}.

La base d'indexation\index{Base d'indexation} ne doit pas recréer la base de données\index{Base de donn\'{e}es} à chaque lancement.\\
La base d'indexation\index{Base d'indexation} ne doit en aucun cas se déconnecter d'elle même, qui plus est sans avoir répondu à toutes les requêtes qu'elle a reçu.\\
Elle doit pouvoir gérer les requêtes des moteur de recherche\index{Moteur de recherche} et d'indexation\index{Moteur d'indexation} ne respectant pas la DTD\index{DTD}. Toutes les flux crées par la base d'indexation\index{Base d'indexation} doivent être conforment à la DTD\index{DTD}.\\
A la réception d'une requête utilisateur, l'opération à effectuer sur la base de données\index{Base de donn\'{e}es} est un \verb"SELECT", en fonction des arguments reçus.\\
Les opérations possibles a effectuer lors de la réception d'un requête venant du moteur d'indexation\index{Moteur d'indexation} sont :
\begin{itemize}
\item nouvelle entrée lors de la création\index{Cr\'{e}ation} d'un fichier\index{Fichier}, avec toutes les dépendances nécessaires, à l'aide de \verb"INSERT"
\item mise à jours d'un champ pour une modification\index{Modification}, en utilisant \verb"UPDATE"
\item suppression\index{Suppression} d'un champ pour une suppression\index{Suppression}, avec \verb"DELETE"
\end{itemize}
Pour ces trois cas, la modification\index{Modification} sur la base de donnée\index{Base de donn\'{e}es} doit se faire de manière complète, c'est à dire sur toutes les entrées correspondantes dans chacune des bases concernées. Autrement dit lors de la suppression d'un fichier il faut parcourir la table des mots pour supprimer toutes les apparitions de ce fichier. Pour la création il faut regarder pour chaque mot à ajouter si il est déjà existant dans la base. Si c'est le cas alors il faut juste rajouter l'identifiant du fichier à la suite des identifiants de fichier, sinon créer une nouvelle entrée pour ce mot avec comme valeur associée l'identifiant du fichier. Dans le cas d'une modification, si le fichier texte de départ était un fichier texte alors il faut faire l'étape de suppression. Si le format d'arrivée est un format texte alors il faut faire l'étape de création.

En cas de coupure de connexion avec un (ou plusieurs) des moteurs, toutes les réponses à envoyer doivent être stockées pendant une certaine durée (10 minutes) dans l'attente que la connexion se refasse.

Un diagramme de classes de la base d'indexation est fourni p.~\pageref{diagramme_classes_bi}.