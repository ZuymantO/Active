\chapter{Introduction}

\section{Les besoins}
Mot(s) du glossaire utilisés : \gls{fichier}.

Le progrès technologique est indéniable à notre époque, notamment en termes de  capacité de stockage des ordinateurs personnels. L'un des enjeux des systèmes pour PC est la gestion des données utilisateur\index{Utilisateur}, à savoir permettre à ces derniers d'accéder rapidement à une information datant de plusieurs semaines, mois ou années sans y perdre des heures de recherche.

La capacité de nos disques durs peut atteindre aujourd'hui des milliers de gigaoctets (Go) et nous pouvons donc stocker des milliards de fichiers\index{Fichier} sur un ordinateur.

Un gestionnaire de système de fichiers\index{Fichier} est un programme permettant d'accéder à ces données, d'y naviguer et de visualiser leur relation (quel fichiers\index{Fichier} appartient à quel dossier\index{Dossier} par exemple). Ce programme n'a pas pour rôle de trier mais de rechercher les fichiers\index{Fichier} et c'est pourquoi, en l'état, l'utilisateur\index{Utilisateur} devra trier son disque dur à la main, si c'est ce qu'il souhaite. Bien que la plupart des systèmes d'exploitation\index{Syst\`{e}me d'exploitation} répandus disposent, en plus du gestionnaire de système de fichiers\index{Fichier}, d'un programme d'indexation\index{Indexation}, nous souhaitons écrire un programme alternatif au programme d'indexation\index{Indexation} natif de Linux\index{Linux} / Unix\index{Unix} (nous n'envisageons pas, du moins dans un premier temps, de développer notre produit pour les systèmes d'exploitation Windows\index{Windows}) et pourquoi ne pas le surpasser sur le plan technique et esthétique.

La réalisation du projet ne doit nécessiter aucun fond d'investissement, user uniquement de technologies open sources ou libres d'utilisation sans contraintes particulières sur le produit final. Pour palier à tout dépense supplémentaire (notamment sur le matériel de travail), nous avons à disposition des machines sous différentes plates-formes (NT, Linux\index{Linux}, Unix\index{Unix}) connectées à internet. Tout module\index{Module} permettant la réalisation du projet est inexistant et devra alors être développé par l'équipe en charge du projet.
Le but est bien entendu de rendre le programme terminé, mais également de parvenir à produire ce dernier par une démarche professionnelle et de manière coopérative sous la direction d'un chef de projet.

\section{Brève description}
Mot(s) du glossaire utilisés : \gls{arborescence}, \gls{fichier}.

On peut distinguer facilement trois modules\index{Module} dans le projet :
\begin{itemize}
\item un moteur de recherche\index{Moteur de rechercher} : ce module\index{Module} est en charge notamment de la communication\index{Communication} avec l'utilisateur\index{Utilisateur}
\item une base d'indexation\index{Base d'indexation} : ce module\index{Module} est en charge de la sauvegarde des données
\item un moteur d'indexation\index{Moteur d'indexation} : ce module\index{Module} est en charge de la surveillance\index{Surveillance} des fichiers\index{Fichier} sur une arborescence\index{Arborescence} donnée.
\end{itemize}

\begin{center}
\includegraphics[scale=0.45]{"images/fonctionnement_global_d'active"}
\captionof{figure}{Fonctionnement global d'Active}
\label{fonctionnement-global}
\end{center}

\section{Objectifs du produit}
Mot(s) du glossaire utilisés : \gls{arborescence}, \gls{fichier}.

Le produit que nous souhaitons développer devra répondre à certaines normes de communications\index{Communication} définies avec d'autres projets afin d'établir une inter-opérabilité\index{Inter-op\'{e}rabilit\'{e}} au niveau des trois modules\index{Module} principaux, définis brièvement dans la section précédente.

Il devra être opérationnel sur plusieurs architectures matériels afin qu'un utilisateur\index{Utilisateur} puisse l'utiliser sur plusieurs systèmes d'exploitation, sans avoir à changer ses habitudes.

Le module\index{Module} \enquote{moteur d'indexation}\index{Moteur d'indexation} sera chargé d'analyser les modifications\index{Modification} sur l'arborescence\index{Arborescence} surveillée et d'en informer le module\index{Module} \enquote{base d'indexation}\index{Base d'indexation} afin que celui-ci mette à jour les informations dans la base de données\index{Base de donn\'{e}es}, qui stocke les informations sur tous les fichiers\index{Fichier} sauvegardés.

L'arborescence\index{Arborescence} surveillée doit être modifiable par l'utilisateur\index{Utilisateur}, ce qui entraîne, à chaque changement, une nouvelle indexation\index{Indexation} de la nouvelle arborescence\index{Arborescence} à surveiller.

Notre produit devra permettre à l'utilisateur\index{Utilisateur} de relancer une indexation\index{Indexation} complète à tout
moment afin de s'assurer du bon contenu de la base de données\index{Base de donn\'{e}es}. Bien évidement, si le
produit est utilisé sur un appareil portable, lors d'une telle demande le moteur d'indexation\index{Moteur d'indexation} devra prendre en compte l'autonomie restante afin de ne pas empêcher l'utilisateur\index{Utilisateur} de pouvoir utiliser son appareil. Dans ce cas un message de confirmation d'indexation\index{Indexation} immédiate devra être donné à l'utilisateur\index{Utilisateur}.

Nous souhaitons que l'utilisateur\index{Utilisateur} puisse faire une recherche avec un maximum d'arguments
afin de limiter au maximum les résultats, ce qui permet de retrouver plus facilement un fichier\index{Fichier}.

L'affichage des résultats devra offrir la possibilité d'être triés.