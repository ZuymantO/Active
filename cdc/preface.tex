\chapter*{Préface}
\section*{Document}
Ce document a pour but de décrire les besoins, les exigences, les contraintes et les limites que nous allons devoir respecter durant la réalisation de notre projet.

Il résume les tâches que le programme doit pouvoir effectuer.

Il est composé de d'une première partie ouverte à tout lecteur et d'une partie technique qui explique les spécificités du produit et est plutôt destinée à un public averti.

Il sert également de planning pour délimiter le temps nécessaire à la réalisation de chaque étape du projet et ce dans le but de le terminer à temps.

\section*{Versions}
Nous souhaitons présenter ci-dessous l'intérêt de chaque version de ce présent document. Les modifications et ajouts importantes.

\subsection*{Version 2}
Finalisation de la spécification en vu du début de développement.

Définition du langage de requête et liste des charges.

\subsection*{Version 2 : Beta $\leftarrow$ (HEAD)}
Définition du langage de requête et support de définition. Choix final de la structure global

Des langages et technologies utilisés.

\subsection*{Version 2 : Alpha}
Restructuration du document. Avec respect du modèle donné en cours

\subsection*{Version 1}
Brouillon complet avec éléments de spécification et de la structure global du projet