\chapter{Annexes}

\section{DTD}\label{dtd}

\subsection{Base d'indexation $\leftrightarrow$ Moteur d'indexation}\label{dtd_bi_mi}
\begin{lstlisting}[frame=single]
<!-- REGLE GENERALE Chaque fichier Xml doit contenir la ligne suivante :
     <?xml version="1.0" encoding="UTF-8"?>
-->

<!ELEMENT INDEXATION (RENOMMAGES?, MODIFICATIONS?, SUPPRESSIONS?, CREATIONS?)>
<!-- comme convenue toutes les balises de l'indexation sont facultatives -->

<!ELEMENT RENOMMAGES (FICHIERRENOMME)*>
<!ELEMENT MODIFICATIONS (FICHIERMODIFIE)*>
<!ELEMENT SUPPRESSIONS (FICHIERSUPPRIME)*>
<!ELEMENT CREATIONS (FICHIERCREE)*>

<!-- un fichier renomme n'est pas necessairement un fichier modifie -->
<!ELEMENT FICHIERRENOMME (PATH, NEWPATH)>
<!-- un fichier modife necessite une re-indexation -->
<!ELEMENT FICHIERMODIFIE (PATH, DATEMODIFICATION, TAILLE, PROPRIETAIRE, GROUPE, PERMISSIONS, INDEXAGE, NEWPATH?)>
<!ELEMENT FICHIERSUPPRIME (PATH)>
<!ELEMENT FICHIERCREE (PATH, format, DATECREATION, TAILLE, PROPRIETAIRE, GROUPE, PERMISSIONS, INDEXAGE)>

<!-- les meta-donnees -->
<!ELEMENT PATH (#PCDATA)>
<!ELEMENT format (#PCDATA)>
<!ELEMENT DATECREATION (#PCDATA)>
<!ELEMENT DATEMODIFICATION (#PCDATA)>
<!ELEMENT TAILLE (#PCDATA)>
<!ELEMENT PROPRIETAIRE (#PCDATA)>
<!ELEMENT GROUPE (#PCDATA)>
<!ELEMENT PERMISSIONS (#PCDATA)>
<!ELEMENT INDEXAGE (MOT*)>
<!ELEMENT MOT (#PCDATA)>
	<!ATTLIST MOT frequence CDATA #REQUIRED>
<!ELEMENT NEWPATH (#PCDATA)>

<!-- les id's seront utlises pour la tracabilitee et la detection d'eventuel erreurs, elle sont tout de meme facultatives -->

<!ATTLIST RENOMMAGES id CDATA #IMPLIED>
<!ATTLIST MODIFICATIONS id CDATA #IMPLIED>
<!ATTLIST SUPPRESSIONS id CDATA #IMPLIED>
<!ATTLIST CREATIONS id CDATA #IMPLIED>
\end{lstlisting}

\newpage
\subsection{Base d'indexation $\leftrightarrow$ Moteur de recherche}

\subsubsection{Recherche}\label{dtd_bi_mr_search}
\begin{lstlisting}[frame=single]
<!-- Description de la requete de recherche -->
<!ELEMENT SEARCH (WORD, CONTENT?, PATHDIR?, PERM?, EXTENSION?, TIMESLOT?)>
	<!ATTLIST SEARCH id CDATA #REQUIRED>
<!-- Le mot a rechercher -->
<!ELEMENT WORD (#PCDATA)>
<!-- Un booleen qui dit si l'on fait une recherche de contenu (true) ou une recherche sur les nom de fichier (false) -->
<!ELEMENT CONTENT (#PCDATA)>
<!-- Le nom du fichier a partir duquel
     on recherche -->
<!ELEMENT PATHDIR (#PCDATA)>
<!-- Les permissions du fichier a chercher -->
<!ELEMENT PERM (#PCDATA)>
<!-- L'extension des fichiers a chercher -->
<!ELEMENT EXTENSION (#PCDATA)>
<!-- Intervalle de temps -->
<!ELEMENT TIMESLOT (BEGIN, END)>
<!-- Debut de l'intervalle -->
<!ELEMENT BEGIN (DAY, MONTH, YEAR)>
<!-- Fin de l'intervalle -->
<!ELEMENT END (DAY, MONTH, YEAR)>
<!-- Le jour -->
<!ELEMENT DAY (#PCDATA)>
<!-- Le mois -->
<!ELEMENT MONTH (#PCDATA)>
<!-- L'annee -->
<!ELEMENT YEAR (#PCDATA)>
\end{lstlisting}

\subsubsection{Résultat}\label{dtd_bi_mr_result}
\begin{lstlisting}[frame=single]
<!-- Balise contenant les resultats de la requete search -->
<!ELEMENT RESULT (FILE*)>
	<!ATTLIST RESULT id CDATA #REQUIRED>
<!-- Balise file correspond a un fichier resultat donc n balises files = n resultats -->
<!ELEMENT FILE (NAME, PATH, PERM, SIZE, LASTMODIF?, PROPRIO?)>
<!-- Le nom du fichier -->
<!ELEMENT NAME (#PCDATA)>
<!-- Le chemin complet du fichier -->
<!ELEMENT PATH (#PCDATA)>
<!-- Les permissions du fichier -->
<!ELEMENT PERM (#PCDATA)>
<!-- La taille du fichier -->
<!ELEMENT SIZE (#PCDATA)>
<!-- La date de derniere modification du fichier -->
<!ELEMENT LASTMODIF (#PCDATA)>
<!-- Le proprietaire du fichier -->
<!ELEMENT PROPRIO (#PCDATA)>
\end{lstlisting}

\section{Moteur de recherche}

\subsection{Cas d'utilisation}\label{utilisation-recherche}
\begin{center}
\includegraphics[scale=0.55]{"images/cas_utilisation_recherche"}
\captionof{figure}{Cas d'utilisation en recherche}
\end{center}

\subsection{Diagramme de séquence}\label{sequence-recherche}
\begin{center}
\includegraphics[scale=0.44]{"images/sequence_util"}
\captionof{figure}{Diagramme de séquence en recherche}
\end{center}

\subsection{Diagramme de classes}\label{diagramme_classes_mr}
\begin{center}
\includegraphics[scale=0.6]{"images/diagramme_classes_mr"}
\captionof{figure}{Diagramme de classes du moteur de recherche}
\end{center}

% pour une meilleure mise en page
\newpage

\section{Moteur d'indexation}

\subsection{Côté daemon}\label{decision-daemon}
\begin{center}
\includegraphics[scale=0.37]{"images/decision_mi"}
\captionof{figure}{Diagramme de décision du moteur d’indexation au niveau du daemon}
\end{center}

\newpage
\subsection{Côté utilisateur}\label{decision-mi}
\begin{center}
\includegraphics[scale=0.45]{"images/decision_util_mi"}
\captionof{figure}{Diagramme de décision du moteur d’indexation en interaction avec l'utilisateur}
\end{center}

% pour une meilleure mise en page
\newpage

\subsection{Diagramme de classes}\label{diagramme_classes_mi}
\begin{center}
\includegraphics[scale=0.4]{"images/diagramme_classes_mi"}
\captionof{figure}{Diagramme de classes du moteur d'indexation}
\end{center}

\section{Base d'indexation}

\subsection{Modélisation de la base de données}\label{bdd}
\begin{center}
\includegraphics[scale=0.6]{"images/modelisation_bdd"}
\captionof{figure}{Modélisation de la base de données}
\end{center}

\subsection{Diagramme de classes}\label{diagramme_classes_bi}
\begin{center}
\includegraphics[scale=0.46]{"images/diagramme_classes_bi"}
\captionof{figure}{Diagramme de classes de la base d'indexation}
\end{center}

\section{Common}\label{common}

\subsection{Diagramme de classes pour \enquote{AQuery}}
\begin{center}
\includegraphics[scale=0.6]{"images/diagramme_classes_aquery"}
\captionof{figure}{Diagramme de classes de la partie relative à \enquote{AQuery}}
\end{center}

\subsection{Diagramme de classes pour la génération du XML}
\begin{center}
\includegraphics[scale=0.6]{"images/diagramme_classes_xmlgeneration"}
\captionof{figure}{Diagramme de classes de la partie relative à la génération XML}
\end{center}

\newpage
\subsection{Diagramme de classes pour le parser XML}
\begin{center}
\includegraphics[scale=0.6]{"images/diagramme_classes_xmlparser"}
\captionof{figure}{Diagramme de classes de la partie relative au parser XML}
\end{center}

% pour une meilleure mise en page
\newpage

\section{Stop list}\label{stop_list}
\subsection{Stop list française}

\textbf{A} : 
alors, 
au, 
aucuns, 
aussi, 
autre, 
avant, 
avec, 
avoir

\textbf{B} :
bon

\textbf{C} :
car, 
ce, 
cela, 
ces, 
ceux, 
chaque, 
ci, 
comme, 
comment, 
ça

\textbf{D} :
dans, 
des, 
du, 
dedans, 
dehors, 
depuis, 
deux, 
devrait, 
doit, 
donc, 
dos, 
droite, 
début

\textbf{E} :
elle, 
elles, 
en, 
encore, 
essai, 
est, 
et, 
eu, 
étaient, 
état, 
étions, 
été, 
être

\textbf{F} :
fait, 
faites, 
fois, 
font, 
force

\textbf{H} :
haut, 
hors

\textbf{I} :
ici, 
il, 
ils

\textbf{J} :
je, 
juste

\textbf{L} :
la, 
le, 
les, 
leur, 
là

\textbf{M} :
ma, 
maintenant, 
mais, 
mes, 
mine, 
moins, 
mon, 
mot, 
même

\textbf{N} :
ni, 
nommés, 
notre, 
nous, 
nouveaux

\textbf{O} :
ou, 
où

\textbf{P} :
par, 
parce, 
parole, 
pas, 
personnes, 
peut, 
peu, 
pièce, 
plupart, 
pour, 
pourquoi

\textbf{Q} :
quand, 
que, 
quel, 
quelle, 
quelles, 
quels, 
qui

\textbf{S} :
sa, 
sans, 
ses, 
seulement, 
si, 
sien, 
son, 
sont, 
sous, 
soyez, 
sujet, 
sur

\textbf{T} :
ta, 
tandis, 
tellement, 
tels, 
tes, 
ton, 
tous, 
tout, 
trop, 
très, 
tu

\textbf{V} :
valeur, 
voie, 
voient, 
vont, 
votre, 
vous, 
vu
% pour une meilleure mise en page
\newpage
\subsection{Stop list anglaise}

\textbf{A} : 
a, 
about, 
above, 
after, 
again, 
against, 
all, 
am, 
an, 
and, 
any, 
are, 
aren't, 
as, 
at

\textbf{B} : 
be, 
because, 
been, 
before, 
being, 
below, 
between, 
both, 
but, 
by

\textbf{C} : 
can't, 
cannot, 
could, 
couldn't

\textbf{D} : 
did, 
didn't, 
do, 
does, 
doesn't, 
doing, 
don't, 
down, 
during

\textbf{E} : 
each

\textbf{F} : 
few, 
for, 
from, 
further

\textbf{H} : 
had, 
hadn't, 
has, 
hasn't, 
have, 
haven't, 
having, 
he, 
he'd, 
he'll, 
he's, 
her, 
here, 
here's, 
hers, 
herself, 
him, 
himself, 
his, 
how, 
how's

\textbf{I} : 
i, 
i'd, 
i'll, 
i'm, 
i've, 
if, 
in, 
into, 
is, 
isn't, 
it, 
it's, 
its, 
itself

\textbf{L} : 
let's

\textbf{M} : 
me, 
more, 
most, 
mustn't, 
my, 
myself

\textbf{N} : 
no, 
nor, 
not

\textbf{P} : 
of, 
off, 
on, 
once, 
only, 
or, 
other, 
ought, 
our, 
ours, 
ourselves, 
out, 
over, 
own

\textbf{S} : 
same, 
shan't, 
she, 
she'd, 
she'll, 
she's, 
should, 
shouldn't, 
so, 
some, 
such

\textbf{T} : 
than, 
that, 
that's, 
the, 
their, 
theirs, 
them, 
themselves, 
then, 
there, 
there's, 
these, 
they, 
they'd, 
they'll, 
they're, 
they've, 
this, 
those, 
through, 
to, 
too

\textbf{U} : 
under, 
until, 
up

\textbf{V} : 
very

\textbf{W} : 
was, 
wasn't, 
we, 
we'd, 
we'll, 
we're, 
we've, 
were, 
weren't, 
what, 
what's, 
when, 
when's, 
where, 
where's, 
which, 
while, 
who, 
who's, 
whom, 
why, 
why's, 
with, 
won't, 
would, 
wouldn't

\textbf{Y} : 
you, 
you'd, 
you'll, 
you're, 
you've, 
your, 
yours, 
yourself, 
yourselves